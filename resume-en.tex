%%
%% Copyright (c) 2022 Artur Sinila <personal@logarithmus.dev>
%% CC BY 4.0 License
%%
%% Résumé
%% ------
%% A short document (1-2 pages) to sum up the job-related accomplishments
%% and experience.
%%

% English version
\documentclass{resume}

% Adjust icon size (default: same size as the text)
% \iconsize{\Large}

% File information shown at the footer of the last page
\fileinfo{%
	\faCopyright{} 2022, Artur Sinila \hspace{0.5em}
	\creativecommons{by}{4.0} \hspace{0.5em}
	\githublink{Logarithmus}{resume} \hspace{0.5em}
	\faEdit{} \today
}

\name{Artur}{Sinila}

\keywords{GNU, Linux, Programming, Rust, C, C++, Shell}

% \tagline{\icon{\faBinoculars}} <position-to-look-for>}
% \tagline{<current-position>}

% \photo{<height>}{<filename>}

\profile{
	\mobile{+380 63 250 52 79}
	\email{personal@logarithmus.dev}
	\birthday{1999 Aug.}
	\address{Kyiv, Ukraine (open for relocation \& remote)} \\
	\website{logarithmus.dev}
	\github{Logarithmus}
	\linkedin{artur-sinila} \\
	\degree{B.Sc undergraduate in Programmable Mobile Systems}
	\university{Belarusian State University of Informatics and Radioelectronics} \\
	% Custom information:
	% \icontext{<icon>}{<text>}
	% \iconlink{<icon>}{<link>}{<text>}
}

\begin{document}
\makeheader

%======================================================================
% Summary & Objectives
%======================================================================

C++, C and Rust programmer. GNU/Linux enthusiast.
Passionate about technology and learning new skills.
Contribute to free and open source software projects.
Like solving challenging problems and optimizing performance.

%======================================================================
\sectionTitle{Work experience}{\faBriefcase}
%======================================================================
\begin{experiences}
	\experience%
		[April 2019]%
		{April 2020}%
		{Software Developer @ Itransition}%
		[\begin{itemize}
			\item Implemented basic Robotic Process Automation (RPA) for JD Edwards ERP system with OpenCV \& Rust
			\item Maintained Oracle database used as a backend for JD Edwards (e.g. fixing broken data records).
			\item Set up RDP session over OpenConnect VPN on Ubuntu and documented the whole process for coworkers.
			\item Communicated in English with Japanese customer.
		\end{itemize}]
\end{experiences}

%======================================================================
\sectionTitle{Skills}{\faWrench}
%======================================================================
\begin{competences}[10em]
	\competence{Operating Systems}{
		\icon{\faLinux} GNU/Linux (using on daily basis for 3 years), \icon{\faWindows} Windows
	}
	\competence{Programming}{%
		Proficient: \textbf{C++} and \textbf{C} (2+ years), \textbf{Rust} (1+ year).
		Familiar: \textbf{Java}, \textbf{Kotlin}
	}
	\competence{Tools}{%
		Git, Vim, Shell, Make, CMake, SSH, GDB, LaTeX
	}
	\competence{Databases}{%
		MariaDB, MySQL
	}
	\competence{Cloud \& Virtualization}{%
		KVM, Digital Ocean, Firebase
	}
	\competence{Embedded systems}{%
		AVR, KiCad, basic soldering
	}
	\competence{Mobile development}{%
		Android
	}
	\competence{\icon{\faLanguage} Languages}{
		\textbf{English} --- Intermediate (B1), %
		\textbf{Ukrainian} --- Beginner (A1), %
		\textbf{Belarusian, Russian} --- native
	}
\end{competences}

%======================================================================
\sectionTitle{Personal Projects}{\faCode}
%======================================================================
\begin{itemize}
	\item \link{https://github.com/Logarithmus/astro-cpp}{\textbf{astro-cpp}}
		(\textbf{C++}):
		Astrodynamics library in C++. Includes elliptic Kepler's equation \& Lambert's problem solvers.
	\item \link{https://github.com/Logarithmus/simple-cpp-programs}{\textbf{simple-cpp-programs}}
		(\textbf{C++}):
		simple programs written to practice C++\\
		(unique digits counter, binary-search number guessing game, multiplication table quiz, quadratic equation solver)
	\item \link{https://github.com/Logarithmus/led-wave-potentiometer}{\textbf{led-wave-potentiometer}}
		(\textbf{C}, \textbf{C++}, \textbf{FreeRTOS}):
		10 blinking LEDs, blinking speed is controlled by the potentiometer and is displayed on the 7-segment display. Employs FreeRTOS for multitasking. Based on ATMega88PA AVR microcontroller.
	\item \link{https://github.com/Logarithmus/battleship}{\textbf{battleship}}
		\textbf{[WIP]} (\textbf{C++, Boost Beast, SFML}):
		Battleship game with multiplayer.
	\item \link{https://github.com/Logarithmus/railway-reservation-rs}{\textbf{railway-reservation-rs}}
		(\textbf{Rust}, \textbf{actix-web}, \textbf{MySQL (MariaDB)}):
		Railway tickets reservation website (university course project).
	\item \link{https://github.com/Graph-Donte-Crypto/AstroGraphicRust}{\textbf{AstroGraphicRust}}
		\textbf{[WIP]} (\textbf{Rust}, \textbf{kiss3d}):
		a project for visualization of interplanetary trajectories with gravity assists
	\item \link{https://github.com/Logarithmus/paint-droid}{\textbf{paint-droid}}
		(\textbf{Android}, \textbf{Kotlin}):
		Paint app supporting choosing pen color \& width, undo/redo, drawing of shapes and text labels.
	\item \link{https://github.com/Logarithmus/quiz-droid}{\textbf{quiz-droid}}
		(\textbf{Android}, \textbf{Kotlin}):
		Quiz Android app, uses Firebase for fetching questions \& answers.
	\item \link{https://github.com/Logarithmus/table-of-users}{\textbf{table-of-users}}
		(\textbf{Java}, \textbf{Spring Boot}, \textbf{Hibernate}):
		Basic CRUD app. OAuth2 via Facebook, Google and GitHub.
	\item \link{https://github.com/Logarithmus/resume}{\textbf{resume}}
		(\textbf{\LaTeX}):
		The template and source files of \emph{this resume}.
\end{itemize}

%======================================================================
\sectionTitle{Contributions}{\faCodeBranch}
%======================================================================
\begin{itemize}
	\item \link{https://github.com/void-linux/void-packages}{\textbf{Void Linux}} (Linux distribution): added new packages (\link{https://github.com/void-linux/void-packages/pull/25159}{\textbf{avr-gdb}}, \link{https://github.com/void-linux/void-packages/pull/25742}{\textbf{simutron}}, \link{https://github.com/void-linux/void-packages/pull/25360}{\textbf{symlinks}}, \link{https://github.com/void-linux/void-packages/pull/25942}{\textbf{avra}}),\\
		updated \& fixed packages (\link{https://github.com/void-linux/void-packages/pull/25147}{\textbf{avrdude}}, \link{https://github.com/void-linux/void-packages/pull/26144}{\textbf{chromium}}, \link{https://github.com/void-linux/void-packages/pull/26347}{\textbf{s3cmd}}, \link{https://github.com/void-linux/void-packages/pull/29096}{\textbf{imv}}, \link{https://github.com/void-linux/void-packages/pull/31667}{\textbf{pulseeffects}}, \link{https://github.com/void-linux/void-packages/pull/31807}{\textbf{rust-analyzer}})
	\item \link{https://github.com/rui314/mold}{\textbf{mold}} (a modern linker): improved docs, optimized CI, helped to fix some tests.
	\item \link{https://gitlab.com/lucciocarreras/sayonara-player}{\textbf{Sayonara}} (music player): fix compilation with \link{https://musl.libc.org}{\textbf{musl}} libc
		\item \link{https://github.com/autozimu/LanguageClient-neovim}{\textbf{LanguageClient-neovim}} (LSP plugin for NeoVIM): added support for rust-analyzer's type \& parameter hints.
		\item \link{https://github.com/sadko4u/lsp-plugins}{\textbf{lsp-plugins}} (audio plugins for Linux): fix compilation with \link{https://musl.libc.org}{\textbf{musl}}, improve debugging experience,\\
		support cross-compilation (waiting for review)
	\item \link{https://github.com/avr-rust/avr-mcu}{\textbf{avr-rust/avr-mcu}}, \link{https://github.com/avr-rust/avrd}{\textbf{avr-rust/avrd}} (AVR device definitions): improve compile times, made code more idiomatic
\end{itemize}

%======================================================================
\sectionTitle{Education}{\faBriefcase}
%======================================================================
\begin{educations}
	\education%
		{September 2017}%
		[June 2021]%
		{Belarusian State University of Informatics and Radioelectronics}%
		{Faculty of Computer-Aided Design}%
		{Programmable Mobile Systems}%
		{B.Sc undergraduate}
\end{educations}

\end{document}
